\section{Anneaux}
  \begin{definition}[Anneau]
    Soit $A$ un ensemble non vide muni de deux lois de composition $+$ et $*$.

    On dit que $(A,+,*)$ est un anneau si: \begin{itemize}
      \item $(A,+)$ est un groupe abélien
      \item $*$ est associative
      \item $*$ est distributive par rapport à $+$
      \item Il existe un élément neutre pour $*$
    \end{itemize}
  \end{definition}

  $(\ZZ,+,\times), (\RR,+,\times)$ sont des anneaux,
  $(\mathcal{M}_n, +,\times)$ aussi, mais non commutatif.


  \begin{definition}[Sous-anneau]
    Soit $(A,+,*)$ un anneau.

    On dit que $(B,+,*)$ est un anneau si: \begin{itemize}
      \item $\emptyset \neq B \subseteq A$
      \item $(B,+)$ est un sous-groupe additif
      \item $B$ est laissé stable par $*$
      \item $e_A \in B$
    \end{itemize}
  \end{definition}


  \begin{definition}[Corps]
    Un anneau $(A,+,*)$ est un corps si:\begin{itemize}
      \item $\#A>2$
      \item $\forall x\in A\backslash \{ 0_+\}, \exists x^{-1} \text{ pour } *$
    \end{itemize}
  \end{definition}


  \begin{definition}[Intégrité]
    Un anneau $(A,+,*)$ est intègre si $x*y=0 \iff x=0\vee y=0$.
  \end{definition}


%Fin du cours 1
