\section{Groupe}
  \begin{definition}[Loi de composition interne]
    Est appelée loi de composition interne toute application $\varphi$:\begin{eqnarray*}
      \varphi : X\times X &\rightarrow& X\\
                (x, y)    &\mapsto& \varphi(x,y)
  \end{eqnarray*}
  \end{definition}

  On insère souvent un symbole à la place de $\varphi$: \[
    \varphi(x,y) = x*y \in X
  \]

  L'addition usuelle est une loi de composition interne dans $\NN,\ZZ,\QQ,\RR,\CC,\RR^k$.

  La composition ($\circ$) est une loi de composition interne sur $\mathcal{F}(E,E)$.

  Soit $X = \mathcal{P}(E)$. L'union ($\cup$), l'intersection ($\cap$) sont des
  lois de composition interne.

  Le produit scalaire $a.b := \begin{cases} \parenthesis{\mathbb{K}^n}^2\to\mathbb{K}\\
                                            (a,b)\mapsto\sum_{i=0}^n a_i b_i
                               \end{cases}$ n'est pas une loi de composition interne.

\begin{definition}[Groupe]
  Soit $G$ un ensemble muni d'une loi de composition interne $*$.

  On dit que $(G,*)$ est un groupe si: \begin{itemize}
    \item $*$ admet un élément neutre
    \item $*$ est associative
    \item Pour tout élément de $G$ il en existe un inverse pour $*$.
  \end{itemize}
\end{definition}

$(\ZZ,+),(\RR,+),(\QQ,+),(\CC,+),(\QQ^*,\times),(\RR^*,\times),(\CC^*,\times),(\rdbrackets{\pm 1},\times)$
sont tous des groupes, mais $(\ZZ^*, \times)$ n'en est pas un.

$0$ n'a jamais d'inverse.

$(\mathcal{B}(E) = \rdbrackets{\text{bijections de $E$}}, \circ)$ est un groupe.

$(\mathbb{D},+)$ est un groupe additif. $(\mathbb{D}\backslash\rdbrackets{0},\times)$ n'en est pas un.

\begin{definition}[Commutativité]
  $*$ sur $X$ non vide est dite commutative si
  \[
    \forall x, y \in X, x*y = y*x
  \]
\end{definition}

\begin{definition}[Groupe abélien]
   Un groupe $(G,*)$ où $*$ est commutative est dit \emph{groupe commutatif} ou
   \emph{groupe abélien}.
\end{definition}

$\parenthesis{\mathcal{GL}\parenthesis{\RR^2}, \circ}$ est non abélien.

\paragraph{Notation additive}
  Lors qu'un groupe $(E,+)$ est abélien, alors il est usuel de noter son
  élément neutre $0$ et les inverses $-x$.

\begin{definition}[Sous-groupe]
  Soit $(G,*)$ un groupe, $H\subseteq G$, $H\neq \emptyset$.

  $(H,*)$ est un sous-groupe de $G$ si $*$ est interne à $H$ et $(H,*)$ est un groupe.
\end{definition}

$(\ZZ,+)$ est un sous-groupe $(\CC,+)$.

\paragraph{Caractérisation}
  Soit $(G,*)$ un groupe, $H\subseteq G$, $H\neq \emptyset$.
  Alors $(H,*)$ est un sous-groupe si $\begin{cases}
    \forall x,y\in H, x*y\in H\\
    \forall x\in H, x^{-1} \in H
  \end{cases}$ ou $\forall x,y \in H, xy^{-1}\in H$.


\begin{proposition}[Unicité de l'élément neutre]
  Soit $(G,*)$ un groupe. Alors il existe un unique élément neutre $e$.
\end{proposition}

\begin{proof}
  \begin{eqnarray*}
    \forall x \in G: x*e=e*x = e'*x = x*e' = x\\
    \begin{cases}
      xe &= xe'\\
      ex &= e'x
    \end{cases} \iff
      x^{-1}xe = x^{-1}xe' \iff e= e'
  \end{eqnarray*}
\end{proof}

\begin{proposition}[Unicité de l'inverse]

    Soit $(G,*)$ un groupe. Alors $\forall x\in G, \exists ! x^{-1} : x*x^{-1}=x^{-1}*x = e$.
\end{proposition}

\begin{definition}[Groupe produit]
  Soient $(G,*)$, $(G', *')$ deux groupes.

  Alors on définit le groupe produit par : $(G\times G', *\times *' = \square)$
\end{definition}

\subsection{Morphisme de groupes}
  \begin{definition}
      Soient $(G,*)$, $(G', *')$ deux groupes.

      On appelle morphisme de $(G,*)$ dans $(G',*')$ une application $\varphi : G \rightarrow G'$ telle que \[
        \forall x,y \in G, \varphi(x*y) = \varphi(x)*'\varphi(y)
      \]
  \end{definition}

  On a alors $\varphi(1_G) = 1_{G'}$ et $\varphi(x^{-1}) = \varphi(x)^{-1}$.
